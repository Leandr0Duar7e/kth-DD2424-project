\documentclass{article}

% if you need to pass options to natbib, use, e.g.:
%     \PassOptionsToPackage{numbers, compress}{natbib}
% before loading neurips_2024


% ready for submission
\usepackage[final]{neurips_2024}


% to compile a preprint version, e.g., for submission to arXiv, add add the
% [preprint] option:
%     \usepackage[preprint]{neurips_2024}


% to compile a camera-ready version, add the [final] option, e.g.:
%     \usepackage[final]{neurips_2024}


% to avoid loading the natbib package, add option nonatbib:
%    \usepackage[nonatbib]{neurips_2024}


\usepackage[utf8]{inputenc} % allow utf-8 input
\usepackage[T1]{fontenc}    % use 8-bit T1 fonts
\usepackage{hyperref}       % hyperlinks
\usepackage{url}            % simple URL typesetting
\usepackage{booktabs}       % professional-quality tables
\usepackage{amsfonts}       % blackboard math symbols
\usepackage{nicefrac}       % compact symbols for 1/2, etc.
\usepackage{microtype}      % microtypography
\usepackage{xcolor}         % colors
\usepackage{graphicx}       % for including images


\title{Comparing ResNet and Vision Transformers: Supervised vs Semi-Supervised Learning for Pet Breed Classification}


% The \author macro works with any number of authors. There are two commands
% used to separate the names and addresses of multiple authors: \And and \AND.
%
% Using \And between authors leaves it to LaTeX to determine where to break the
% lines. Using \AND forces a line break at that point. So, if LaTeX puts 3 of 4
% authors names on the first line, and the last on the second line, try using
% \AND instead of \And before the third author name.


\author{%
  Francesco Olivieri \\
  KTH Royal Institute of Technology \\
  \texttt{olivieri@kth.se} \\
  \And
  Inês Mesquita \\
  KTH Royal Institute of Technology \\
  \texttt{inesm@kth.se} \\
  \And
  Leandro Duarte \\
  KTH Royal Institute of Technology \\
  \texttt{ldr0@kth.se} \\
}


\begin{document}


\maketitle


\begin{abstract}
% Brief overview of the project, approach, and key results
% Should be no more than 300 words
\end{abstract}


\section{Introduction}
% Describe the problem and its importance
% Briefly describe what you did and give an overview of results
% Why is transfer learning important?
% Why compare ResNet and ViT?
% Why study semi-supervised learning?


\section{Related Work}
% Discuss published work related to:
% 1. Transfer learning with CNNs and Vision Transformers
% 2. Semi-supervised learning techniques
% 3. Pet breed classification or similar fine-grained classification tasks


\section{Data}
% Describe the Oxford-IIIT Pet Dataset
% How much data, what kind of images
% Data preprocessing and augmentation
% Data splits for supervised and semi-supervised experiments
% How was data reduced for the semi-supervised experiments


\section{Methods}
% Describe your approaches in detail:


\subsection{Models}
% ResNet architecture and configuration
% Vision Transformer architecture and configuration


\subsection{Transfer Learning Approaches}
% Fine-tuning strategies for ResNet
% Fine-tuning strategies for Vision Transformer
% Hyperparameters and optimization details


\subsection{Semi-Supervised Learning}
% Description of the semi-supervised learning technique(s) used
% Implementation details


\section{Experiments}
% Detailed description of experiments and results


\subsection{Binary Classification}
% Results from dog vs cat classification


\subsection{Multi-Class Classification with Full Supervision}
% Results from 37-class breed classification with full supervision
% Comparison of fine-tuning strategies
% Comparison of ResNet vs ViT


\subsection{Multi-Class Classification with Semi-Supervision}
% Results from semi-supervised learning experiments
% Performance with varying amounts of labeled data (50%, 10%, 1%)
% Comparison between ResNet and ViT in semi-supervised setting


\subsection{Ablation Studies}
% Effect of different components of your approach
% Learning rate strategies, regularization, etc.


\section{Conclusion}
% Summary of key findings
% Limitations of the study
% Suggestions for future work


\section{Broader Impacts}
% Discuss potential positive societal impacts
% Discuss potential negative societal impacts
% Consider fairness, privacy, security implications


\section*{References}
% List all references


\appendix
\section{Additional Experiments}
% Place for any additional results that didn't fit in the main paper


\section{Implementation Details}
% More detailed information about implementations
% Hyperparameter settings, training configurations


\end{document}